\newpage

\chapter*{Sažetak}
\addcontentsline{toc}{chapter}{Sažetak}
Pojavom jeftinih mikroupravljača i mini računala mogućnosti povezivanja
fizikalnog svijeta s digitalnim postaju sve veće.
Svrha diplomskog rada je ispitivanje mogućnost korištenja jeftinih mikroupravljača
za jednostavne poslove automatizacije kao i udaljenog upravljanja postrojenja
preko \emph{socket} \emph{servera}.

Za mikroupravljač \emph{Arduino Uno} preko USB-serijskog sučelja i preko \emph{Firmata}
protokola komunicira s računalom. Na njega je je vodena pumpa i senzor koji mjeri
razinu vode u boci te sustav od dvije boce. Na računalu se nalazi \emph{socket} \emph{server}
pisan u \emph{Haskellu} koji upravlja mikroupravljačem. Testni klijent pisan u
\emph{Pythonu} spaja se na \emph{server} te dohvaća mjernu veličinu
i prikazuje trenutno stanje sustava.

Rad je podijeljen u dva dijela: teorijski dio koji detaljno opisuje proces razvoja socket
servera te praktični dio koji se bavi implementacijom postrojenja, regulatora,
\emph{socket servera} i popratnog klijenta.
\\

\noindent\textbf{Ključne riječi:} Haskell, Arduino, Automatizacija,
Python, Socket server, Firmata
\chapter*{Abstract}
{\large\bfseries\MakeUppercase{A socket server for data transmission from a manufacturing plant}}
\\

The rise of cheap microcontrollers and mini computers have enabled new possibilities
of connecting the physical world with the digital.
The purpose of this paper is to explore the possibilities of cheap microcontrollers
for simple automation as well as to enable remote control using a socket server.

The microcontroller Arduino Uno was chosen which communicates through the USB
interface and the  Firmata protocol with the computer. A water pump and a sensor,
which measures the water level, were connected as well as two bottles.
The computer is equipped with a socket server written in Haskell
which controls the microcontroller. The test client written in Python is being
connected to the server and pumps the requested amount of water and shows the
current system status.

The paper is divided into two parts. The first part shall elaborate the theoretical presuppositions
regarding the subject matter. The second shall address the implementation of the system,
the regulators, the socket servera and the accompanying client.
\\

\noindent\textbf{Keywords} Haskell, Arduino, Automation,
Python, Socket server, Firmata
