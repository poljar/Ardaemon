\newpage
\chapter{Implementation}

U ovom poglavlju je objašnjena implementacija socket servera, odgovarajućeg
klijenta i na kraju samog postrojenja.

\newpage
\section{Server}

Arhitektura servera.

\begin{figure}[H]
\centering
\begin{tikzpicture}[every node/.style = {font=\bf},
                    connection/.style = {Latex-Latex, Peach, line width=2}]
    \node[draw=OliveGreen,rectangle, rounded corners,
          fill=OliveGreen, minimum size=100]
          (regulator) {\textcolor{white}{Regulacijska nit}};

    \node[draw=MidnightBlue, rectangle, rounded corners, minimum size=100,
          right=2cm of regulator, fill=MidnightBlue]
          (server) {\textcolor{white}{Posrednička nit}};

    \node[draw=NavyBlue, rectangle, rounded corners,
          minimum size=60, below left=2cm and -2.3cm of server,
          fill=NavyBlue]
          (clientA) {\textcolor{white}{Klijentska nit A}};

    \node[draw=NavyBlue, rectangle, rounded corners,
          minimum size=60, right=of clientA,
          fill=NavyBlue]
          (clientB) {\textcolor{white}{Klijentska nit B}};

    \draw[connection] (regulator) to (server);
    \draw[connection] (server)    to (clientA);
    \draw[connection] (server)    to (clientB);
\end{tikzpicture}
\caption{Arhitektura servera}
\label{fig:architecture}
\end{figure}

Na slici \ref{fig:architecture} je prikazana komunikacija između pojedinih niti
unutar servera. Niti komuniciraju pomoću djeljenih varijabli.
... je dobro.

\begin{listing}[H]
\centering
\begin{minted}[frame=single]{haskell}
startDaemon :: Integer -> FilePath -> Bool -> IO ()
startDaemon port arduinoPort simulate = do
    sock <- socket AF_INET Stream 0

    bindSocket sock $ SockAddrInet (fromInteger port) iNADDR_ANY
    listen sock 50

    pv <- newMVar 0
    referenceChan <- newChan
    let com = ProcCom pv referenceChan

    _ <- forkIO $ mainLoop sock com

    controllerBroker arduinoPort simulate com
\end{minted}
\caption{Main entry point}
\label{lst:main}
\end{listing}

Na ispisu koda \ref{lst:main} je.

\begin{listing}[H]
\centering
\begin{minted}[frame=single]{haskell}
mainLoop :: Socket -> ProcCom PVType -> IO ()
mainLoop sock com = do
    (hdl, host, port) <- accept sock
    let hostinfo = host ++ ":" ++ show port
    noticeM rootLoggerName $ "Connected:    " ++ hostinfo
    _ <- forkIO $ runConn hdl com hostinfo

    mainLoop sock com
\end{minted}
\caption{Server mainloop}
\label{lst:mainloop}
\end{listing}

\begin{listing}[H]
\centering
\begin{minted}[frame=single]{haskell}
runConn :: Handle -> ProcCom PVType -> String -> IO ()
runConn hdl com hostinfo = do
    isEof <- hIsEOF hdl

    if isEof then do
        hClose hdl
        noticeM rootLoggerName $ "Disconnected: " ++ hostinfo
    else do
        contents <- hGetLine hdl
        debugM rootLoggerName $ "RPC request : " ++ contents

        response <- handleMsg com $ C.pack contents
        debugM rootLoggerName $ "RPC response: " ++ C.unpack response

        C.hPutStrLn hdl response
\end{minted}
\caption{Client handler thread}
\label{lst:client handler}
\end{listing}

\begin{listing}[H]
\centering
\begin{minted}[frame=single]{haskell}
controllerBroker :: FilePath -> Bool -> ProcCom PVType -> IO ()
controllerBroker arduinoPort simulate (ProcCom pvMVar refChan) = do
    refMVar <- newMVar 0

    _ <- forkIO $ forever $ do
            ref <- readChan refChan
            swapMVar refMVar ref

    if simulate then
        simulatorLoop refMVar pvMVar
    else do
        controlLoop arduinoPort refMVar pvMVar
        shutDownArduino arduinoPort

    noticeM rootLoggerName "Shutting daemon down."
\end{minted}
\caption{Controler broker}
\label{lst:controler broker}
\end{listing}

\subsection{Regulator}

\begin{listing}[H]
\centering
\begin{minted}[frame=single]{haskell}
forever $ do
    integral  <- liftIO $ takeMVar integralMVar
    reference <- liftIO $ readMVar refMVar

    sensorValue <- analogRead sensor
    let fillHeight = sensorValueFunc sensorValue
    _ <- liftIO $ swapMVar pvMVar fillHeight

    let error = reference - fillHeight

    let proportional_term = kp * error
    let integral_term     = integral + ki * error / sampleTime

    let output = clampAndScaleOutput $ proportional_term + integral_term

    analogWrite pwm_pin output
    delay sampleTime
\end{minted}
\caption{Regulacijska petlja}
\label{lst:regulator}
\end{listing}

\newpage
\section{Client}
\begin{listing}[H]
\centering
\begin{minted}[frame=single]{python}
def main():
    sock = socket.socket(socket.AF_INET, socket.SOCK_STREAM)
    sock.connect((args.host, args.port))

    tank = TankWidget()
    command_line = CommandLine(cmd_list, remote_cmds, sock)
    top = urwid.Frame(tank, None, command_line, 'footer')

    evl = urwid.AsyncioEventLoop(loop=asyncio.get_event_loop())
    loop = urwid.MainLoop(top, palette, event_loop=evl)

    loop.watch_file(sock.fileno(), read_cb)
    loop.set_alarm_in(0.1, periodic_tasks)

    loop.run()
\end{minted}
\caption{Client main loop}
\label{lst:client main}
\end{listing}

\begin{listing}[H]
\centering
\begin{minted}[frame=single]{python}
    def periodic_tasks(loop, data):
        request , _ = lvl_cmd('update-tank', [])
        try:
            sock.send(bytes(request, 'utf-8'))
        except BrokenPipeError as e:
            pass

        tank.update()

        loop.set_alarm_in(args.interval, periodic_tasks)

    loop.watch_file(sock.fileno(), read_cb)
\end{minted}
\caption{Periodic update dings}
\label{lst:periodic tasks}
\end{listing}


\newpage
\newpage
\section{Postrojenje}

\begin{figure}[H]
\centering
\includegraphics[scale=0.80]{figures/shield_schematic.pdf}
\caption{Električna shema postrojenja}
\end{figure}

\newpage
\subsection{Matematički model postrojenja}

\begin{equation} \frac{dV}{dt} = q_{in} - q_{out} \end{equation}

\begin{equation} q_{out} = a \sqrt{2gh} \end{equation}

\begin{equation} q_{in} = k_p V \end{equation}

\begin{equation} A\frac{dh}{dt} = k_p V - a \sqrt{2gh} \end{equation}

\begin{equation} \frac{d\delta h}{dt} = \frac{k_p}{A} \delta V -
                 \frac{a \sqrt{2g}}{2 A \sqrt{h_0}} \delta h \end{equation}

\begin{equation} \frac{d\delta h}{dt} = k_1 \delta V
                 - k_2 \delta h \end{equation}

\begin{equation} s H(s) = k_1 V(s) - k_2 H(s) \end{equation}

\begin{equation} H(s)(s+k_2) = k_1 V(s) \end{equation}

\begin{equation} G(s) = \frac{H(s)}{V(s)} = \frac{k_1}{s+k_2} \end{equation}

\begin{figure}[H]
\centering
\begin{tikzpicture}[every node/.style = {font = \footnotesize}]
    \tikzset{weird fill/.style={append after command={
    \pgfextra
        \draw [#1, sharp corners, fill=#1]%
        (\tikzlastnode.west)%
        [rounded corners=0] |- (\tikzlastnode.north)%
        [rounded corners=3] -| (\tikzlastnode.east)%
        [rounded corners=0] |- (\tikzlastnode.south)%
        [rounded corners=0] -| (\tikzlastnode.west);
        \endpgfextra}}}

    \draw [rounded corners=3] (-5,10) -- (-2,10) -- (-2,9);
    \draw [rounded corners=3] (-5,10.5) -- (-1.5,10.5) -- (-1.5,9);
    \draw (-1.75, 9) circle [x radius=0.25, y radius=0.05];
    \draw (-4, 10.25) node [font=\normalsize] {$q_{in}$};
    \draw [-Latex, thick] (-3.5, 10.25) -- (-2.5, 10.25);

    \draw (0, 2.5) node[rectangle, minimum width=228, minimum height=140,
    fill=NavyBlue] {};

    \draw [fill=NavyBlue] (0,0) circle [x radius=4, y radius=0.5];
    \draw [fill=NavyBlue, thick, dashed] (0,5) circle [x radius=4, y radius=0.5];

    \node [font=\large] (area) {\textcolor{white}{A}};
    \node [font=\large, above=4.4 of area] {\textcolor{white}{V}};

    \draw (4,9)  arc[x radius = 4, y radius = 0.5, start angle=0, end angle=112];
    \draw (4,9)  arc[x radius = 4, y radius = 0.5, start angle=0, end angle=-240];

    \draw (-4,0)  -- (-4,9);
    \draw (4,0)   -- (4,0.5);
    \draw (4,1.2) -- (4,9);

    \draw (4.9, 0.85)
    node[weird fill=NavyBlue, rectangle, minimum width=62, minimum height=19]
    (qout1) {};
    \draw (4.5, 0.85) node [font=\normalsize] {\textcolor{white}{$q_{out}$}};

    \node [below right=-0.1 and -0.7 of qout1, fill=NavyBlue, minimum width=20, minimum
    height=17] (qout2) {};

    \draw [NavyBlue, fill=NavyBlue] (5.2, 0.6) -- (5.2,0.7) -- (5.5,0.7) -- (5.30, 0.4);
    \draw [NavyBlue, fill=NavyBlue] (5.2, 0.6) -- (5.2,0.5) -- (5.5,0.49);
    \draw [rounded corners=3] (4,1.2) -- (6,1.2) -- (6,0);
    \draw [rounded corners=3] (4,0.5) -- (5.3,0.5) -- (5.3,0);

    \draw [fill=NavyBlue] (5.65,0) circle [x radius=0.35, y radius=0.05];

    \draw (-4, 0) -- (-4.5, 0);
    \draw (-4, 5) -- (-4.5, 5);
    \draw [Latex-Latex] (-4.5, 0) -- (-4.5, 5);
    \draw (-4.7, 2.5) node [font=\large] {h};

\end{tikzpicture}
\caption{Shematski model postrojenja}
\end{figure}
