\newpage
\chapter{Zaključak}

U radu je razvijen i predstavljen socket server za komunikaciju s postrojenjem.
Server komunicira s Arduino mikroupravljačkom pločicom i upravlja postrojenjem
koje je spojeno na pločicu. Demonstrirano je uspješno automatsko upravljanje
razine vode u spremniku. Razvijen je i klijent koji u stvarnom vremenu daje uvid
u stanje postrojenja te omogućuje udaljenu promjenu regulirane veličine. Tijekom
izrade, glavni problemi bili su omogućavanje istovremenog obavljanja poslova
regulacije i posluživanja podataka udaljenim klijentima, izrada makete
postrojenja te povezivanje svih dijelova rada u cjelinu.

Predstavljeno riješenje koje koristi regulator unutar socket servera ima svoje
prednosti i nedostatke. Nedostatci su vezani uz brzinu regulacije, zbog
serijskog protokola koji je korišten za komunikaciju postrojenja i socket
servera nije moguće regulirati procese koji imaju jako ograničene vremenske
zahtjeve. Zahtjev za dodatnim računalom pored mikroupravljača za obavljanje
posla regulacije također je jedan nedostatak. Prednosti su mogućnost rapidnog
razvoja regulatora, mogućnost izmjene regulatora bez ponovnog programiranja
mikroupravljača, mogućnost udaljenog upravljanja postrojenjem te fleksibilnost.

Široka dostupnost jeftinih mikroupravljača te malih računalnih platformi
omogućuje korištenje predstavljenog rješenja u razne svrhe kućne
automatizacije. Moguća je izrada mreže automatizirnih procesa kojima je
omogućeno daljinsko upravljanje pomoću socket servera.

Riješenje također ima visoki potencijal za poboljšanja. Moguće je poboljšati
parametre regulatora te ostvariti bolje upravljanje, prebaciti regulator na
mikroupravljač te tako smanjiti latenciju. Na serveru se lako mogu
implementirati razne vrste regulatora te bi se one mogle dinamički izmjenjivati.
Komunikacija klijenta i servera izvršava se nekriptiran. Na serveru
se također može implementirati zahtjev za autentikacijom prije nego se dozvoli uvid u
stanje postrojenja.
