\newpage
\section{Uvod}

Cilj diplomskog rada je ispitati mogućnost korištenja jeftinih mikroupravljača
za jednostavne poslove automatizacije te omogućiti udaljeno upravljanje
postrojenjem preko \emph{socket} \emph{server-a}.

Kako bi se utvrdio cilj diplomskog rada pomoću mikroupravljača je simulirano
postrojenje. Komunikacija prema računalu je ostvarena putem \emph{USB}
sučelja. Računalo šalje naredbe prema mikroupravljaču pomoću komunikacijskog
protokola. \emph{Socket} \emph{server} je izrađen tako da s jedne strane koristi
navedeni protokol da bi komunicirao s mikroupravljačem, a s druge strane
omogućuje klijentima upit u stanje postrojenja i postavljanje referentne
varijable postrojenja.

Za mikroupravljač je odabran \emph{Arduino Uno} koji preko USB-serijskog sučelja
i preko \emph{Firmata} protokola komunicira s računalom. Na \emph{Arduino} je
spojena vodena pumpa i senzor koji mjeri razinu vode u boci. Sustav od dvije
boce čini postrojenje, u jednoj boci održava se razina vode. Na računalu se
nalazi \emph{socket} \emph{server} pisan u \emph{Haskell-u} koji upravlja
mikroupravljačem. Testni klijent pisan u \emph{Python-u} se spaja na
\emph{server} te dohvaća mjernu veličinu (razinu vode u boci) i prikazuje
trenutno stanje sustava.

Rad je podijeljen u dva dijela. U prvom dijelu razrađene su teorijske
pretpostavke vezane uz temu. Važno je napomenuti kako je u ovom dijelu detaljno
opisan proces automatizacije i razvoj socket servera. Drugi dio bavi se
implementacijom postrojenja, regulatora, \emph{socket server-a} i popratnog
klijenta.
